\pdfminorversion=4
\documentclass[utf8]{beamer}
\usepackage{etex}
\usepackage[ngerman]{babel}
\usepackage[utf8]{inputenc}
\usepackage{amsfonts}
\usepackage{amssymb}
\usepackage{amstext}
\usepackage{mathtools}
\usepackage{mathdesign}
\usepackage{tabularx}
\usepackage{tikz}

\usepackage[locale=DE]{siunitx}
\sisetup{
    group-digits=true,
    group-separator={\,},
    group-minimum-digits = 3
}


\usetheme{tib}
\DeclareMathOperator{\sinc}{sinc}

\usetikzlibrary{arrows,snakes,backgrounds,patterns,matrix,shapes,fit,calc,shadows,plotmarks}

\title{Kolloquium zur Masterarbeit}
\subtitle{Entwicklung eines Web-basierten Verfahrens zum Lernen von visuellen Konzepten}
\author[M. Springstein]{Matthias Springstein}
\date{\today}

%\usetheme{Berkeley}
\begin{document}

{
\setbeamertemplate{footline}{}

\begin{frame}
    \titlepage
\end{frame}
}

\begin{frame}
    \frametitle{Gliederung}
    \tableofcontents
\end{frame}

\section{Motivation und Zielsetzung}
\begin{frame}
    \frametitle{Gliederung}
    \tableofcontents[currentsection]
\end{frame}

\subsection{}
\begin{frame}
    \frametitle{Motivation}
    \begin{itemize}
        \item Verringerung des menschlichen Aufwands für die Erstellung eines Trainingsdatensatzes
        \item Nutzung von Webinhalten zur Auswahl von negativen und positiven Trainingsbildern
        \item Automatische Erstellung von Klassifikatoren für die Konzeptdetektion
        \item Verwendung von aktuellen Verfahren zur Klassifikation von Bildern
    \end{itemize}
\end{frame}

\begin{frame}
    \frametitle{Zielsetzung}
    \begin{itemize}
        \item Entwicklung eines System zum Finden von geeigneten Trainingsbeispielen aus dem Internet
        \item Nutzung von Deep Learning Verfahren
        \item Validierung des Systems und des erzeugten Klassifikators
    \end{itemize}
\end{frame}

\section{Stand der Technik}

\begin{frame}
    \frametitle{Gliederung}
    \tableofcontents[currentsection]
\end{frame}

\subsection{Convolutional Neural Network}
\begin{frame}
    \frametitle{Convolutional Neural Network}
    \begin{itemize}
        \item Stand der Technik in mehreren Gebieten des maschinellen Lernens
        \item Nutzung von GPU zum Training des Netzwerkes
        \item Automatisches Erlernen geeigneter Merkmalsextraktoren
    \end{itemize}
    \begin{figure}

    \end{figure}
\end{frame}

\begin{frame}
    \frametitle{Convolutional Neural Network}
    \begin{figure}

    \end{figure}
    \footnotetext{\tiny ICML 2013 tutorial on Deep Learning}
\end{frame}

\section[Verfahrensbeschreibung]{Vorstellung des Verfahrens}

\begin{frame}
    \frametitle{Gliederung}
    \tableofcontents[currentsection]
\end{frame}

\subsection{Konzept}
\begin{frame}
    \frametitle{Konzept}
    \begin{columns}[c]
        \column[c]{0.8\linewidth}
            \begin{itemize}
                \item Download geeigneter Trainigsbildern aus dem Internet
                \item Filterung der Eingangsbilder und Entfernen von Spam
                \item Training eines \emph{Convolutional Neural Network} mit den gefundenen Beispielen
            \end{itemize}
        \column{0.2\linewidth}
        \begin{figure}

        \end{figure}
    \end{columns}
\end{frame}

\subsection{Verarbeitungsschritte}
% \begin{frame}
%   \frametitle{Textueller Filter}
%   \begin{itemize}
%       \item Erzeugen eines Merkmalsvektors, unter Einsatz des Tf-idf-Maßes
%       \item Clusterung der Kandidaten anhand der extrahierten Merkmalsvektoren
%       \item Entfernung aller Cluster mit wenigen Elementen und einem geringen Silhouettenkoeffizient
%       \item Ranking der verbleibenden Cluster, anhand des Abstandes zu einem erstellten Referenzvektor
%   \end{itemize}
% \end{frame}


\begin{frame}
    \frametitle{Visueller Filter}
            \begin{itemize}
                \item Nutzung von \emph{Convolutional Neural Network} zur Merkmalsextraktion
                \item Clusterung der Kandidaten anhand der extrahierten Merkmalsvektoren
                \item Entfernung aller Cluster mit wenigen Elementen und einem geringen Silhouettenkoeffizient
                \item Ranking der verbleibenden Cluster, mithilfe der Distanz der Elemente zum Clusterzentrum
            \end{itemize}
\end{frame}

\begin{frame}
    \frametitle{Visueller Filter}
    \begin{figure}
        \caption{Zwei Cluster des visuellen Filters}
    \end{figure}
\end{frame}

\begin{frame}
  \frametitle{Training des neuronalen Netzes}
  \begin{itemize}
    \item Anzahl der Neuronen in der Ausgangsschicht wird auf die Anzahl der gesuchten Konzepte gesetzt
    \item Training mit dem erstellten Datensatz und dem stochastischen Gradientenverfahren
%    \item Untersuchung verschiedener Netztopologien und Gewichtsinitalisierungen
  \end{itemize}
  \begin{figure}
  \end{figure}
\end{frame}

\section{Ergebnisse}

\begin{frame}
    \frametitle{Gliederung}
    \tableofcontents[currentsection]
\end{frame}

\subsection{Testdatensatz}

\begin{frame}
    \frametitle{Pascal VOC 2012 Datensatz}
    \begin{figure}
        \begin{itemize}
          \item Beinhaltet \num{17125} Trainings- und Validierungsbilder für den Wettbewerb
          \item Nutzung des Validierungsdatensatz der Klassifikationsaufgabe zum Test des Systems
          \item Entfernung aller Bilder die im Trainingsdatensatz und im Validierungsdatensatz enthalten sind
        \end{itemize}

    \end{figure}
\end{frame}


\subsection{Experimentelle Ergebnisse}

\begin{frame}
    \frametitle{Validierung der Spam-Detektion}
    \begin{figure}

    \end{figure}
\end{frame}

\begin{frame}
    \frametitle{Test des inkrementellen Lernansatzes}
    \begin{figure}

    \end{figure}
\end{frame}

\begin{frame}
    \frametitle{Validierung der Nutzung von vortrainierten neuronalen Netzen}
    \begin{figure}

    \end{figure}
\end{frame}

\section[Abschluss]{Zusammenfassung und Ausblick}

\begin{frame}
    \frametitle{Gliederung}
    \tableofcontents[currentsection]
\end{frame}

\subsection{}

\begin{frame}
    \frametitle{Zusammenfassung}
    \begin{center}
        \begin{itemize}
          \item Das System ist in der Lage Bilder herunterzuladen, zu Filtern und ein Klassifikator zu trainieren
          \item Das Filtersystem erhöht die Leistungsfähigkeit des Systems
          \item Durch das Hinzufügen von Trainingsbeispielen erhöht sich die Genauigkeit der einzelnen Klassen
          \item Der Einsatz von vortrainierte Netzen reduziert die Trainingszeit und reduziert den Fehler auf dem Datensatz
        \end{itemize}
    \end{center}
\end{frame}

\begin{frame}
    \frametitle{Ausblick}
    \begin{center}
        \begin{itemize}
          \item Erweiterung des Systems zur Lokalisierung des gesuchten Objektes
          \item Hinzufügen von Quellen, wie Google Bildersuche
          \item Verbesserung des Filtersystems, zum Beispiel durch den Einsatz andere Clusterverfahren
          \item Beschleunigung des Lernverfahrens durch neue Verfahren, wie Batch Normalization
        \end{itemize}
    \end{center}
\end{frame}

\section*{}
\begin{frame}
    \begin{center}
        \Huge Vielen Dank.
    \end{center}
\end{frame}

\section*{}

\begin{frame}
  \frametitle{Textueller Filter}
  \begin{itemize}
      \item Erzeugen eines Merkmalsvektors, unter Einsatz des Tf-idf-Maßes
      \item Clusterung der Kandidaten anhand der extrahierten Merkmalsvektoren
      \item Entfernung aller Cluster mit wenigen Elementen und einem geringen Silhouettenkoeffizient
      \item Ranking der verbleibenden Cluster, anhand des Abstandes zu einem erstellten Referenzvektor
  \end{itemize}
\end{frame}

\end{document}
